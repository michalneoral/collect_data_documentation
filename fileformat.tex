\chap Data Structure

For the purpose of the experiment is good data processed offline. It is therefore important to store the measured data to data structure and then in MATLAB calculate the parameters that are important for the experiment~(chap.~\ref[chap:dataproc]). If the results of experiment are good and quick, will be the calculation in the future transformed from MATLAB to the ROS.


\sec Format of Recorded Data

Data is stored by using rosbag tool (chap.~\ref[secc:rosintro]) in the format ".bag" to the folder set in the "local_options.py" file \urlnote{path_to_workspace/clopema_cvut/clopema_collect_model_data/src/local_options.py}.

\label[sec:topics]
\sec Topics

The CloPeMa robot can produce over two hundred topics (chap.~\ref[secc:rosintro]) when running. Due to the saving disk space and capacity of the transmission channel are recorded only topics which are important to the evaluation of the exporiment. Selected topics are set in "topics.txt"~\urlnote{path_to_workspace/clopema_cvut/clopema_collect_model_data/matlab/topics/topics.txt} and contains these choosen topics:

\begtt
/joint_states
/tf
/xtion1/depth/camera_info
/xtion1/depth_registered/camera_info
/xtion1/rgb/camera_info
/xtion1/depth/image_raw
/xtion1/rgb/image_raw
/feedback_states
\endtt

\sec Measured Data Set of the Garments


\secc Structure of Data Set

\mind{Zde bude popsána datová sada a kde bude uložena}


\secc Format of Names of Recorded Files

Recorded files are stored under different names accord to the form "name_speed_AX.bag" (table~\ref[explanation]). 

\midinsert \clabel[explanation]{Explanation of format file name.}
\ctable{rl}{
name & choosen file name by user\cr
speed & choosen speed of manipulator\cr
A & axis, which was executed movement {\bf R} or {\bf B} (figure~\ref[fig:motomanAxis])\cr
X & number of topics file\cr
}
\caption/t Explanation of format file name.
\endinsert


\secc Description of the Garments

\mind{Zde bude výčet některých použitých látek jako hmotnosti, rozměry ...}

