\chap Conclusion

%Seznamte se s postupem vytváření dynamických fyzikálních modelů látky.
In the introduction of this thesis I describe the methods of obtaining parameters of dynamic model of the garment, based on the mechanical stressing of the garment. These methods are not suitable in our case, so I developed procedure of.

%Seznamte se s možnostmi měření a realizace experimentů na robotickém pracovišti projektu CloPeMa.


%Navrhněte metodu měření a získání obrazových příznaků, které lze použít pro odhad parametrů dynamického fyzikálního modelu látky.


%Implementujte navrženou metodu v prostředí Matlab a potřebné části v prostředí ROS.


%Připravte datovou sadu, která bude použita při ověřování metod odhadování parametrů fyzikálního modelu


%Vše pečlivě zdokumentujte. 









%We met the process of creating dynamic physical models of the garment and obtaining the parameters of the real garment. 

%We design own methods of extraction image features, which can be use to estimate the parameters of the dynamic model of garment. These methods are based on extraction features from RGB images and depthmaps. 

%The CloPeMa robotic manipulator was used for the realization of the experiments. Parts of the code that are responsible for controlling the robot and storage of measured data was written in the ROS environment. 

%Image processing method of depth and RGB cameras were implemented in the MATLAB environment.


%The data set of 46 measurements was prepared to validate the methods for estimation of the model parameters.

%		\Green
%		\begitems \style o
%			* V práci je: ...
%			* Hlavní úspěchy jsou: ...
%			* Důležitými výsledky jsou: ...
%			* Podařilo se: ...
%			* Za nejdůležitější výsledek považuji:
%			* Možnost vynechání kapitoly DISCUSSION a uvedení jejího obsahu sem
%			* Pohled do budoucna (přeformulovat, změnit, rozšířit):
%			\begitems \style o
%				* V případě, že se ukáže tento způsob sběru dat a tvorba modelu (odkaz na jinou bc.práci) užitečnou, bylo by dobré naprogramovat celý tento postup i s tvorbou modelu v operačním systému ROS, aby nebylo třeba dalších výpočetních nástrojů (MATLAB).
%			\enditems
%			* Rekapitulovat naplnění všech bodů práce
%		\enditems
%		\Black

%\mind{do 11.5.}


