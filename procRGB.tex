\label[sec:rgb]
\sec Extraction of Features from RGB images

The sequence of RGB images is captured with the xtion camera (chap.~\ref[secc:camera]). Images have resolution 640$\times$480~px (width$\times$height). RGB images are stored in the "bagfile"s like single row vector. After loading data into MATLAB~(chap.~\ref[sec:loadToMatlab]) data are converted from row vector to matrix ${\bf I}_{R,G,B}(u,v)$ of RGB images with dimensions 480$\times$640~px ($u\times v$). Raw RGB data shows figure~\ref[fig:extRGB].
\Black

\medskip \clabel[fig:extRGB]{Sample RAW RGB images}
\picw=18cm 
%\centerline {\inspic pictures/todo.pdf \hfil\hfil \inspic pictures/todo.pdf }\nobreak
%\centerline {a)\hfil\hfil b)}\nobreak\medskip
%\centerline {\inspic pictures/todo.pdf \hfil\hfil \inspic pictures/todo.pdf }\nobreak
%\centerline {c)\hfil\hfil d)}\nobreak\medskip
\cinspic pictures/rgbsample.pdf
\caption/f Sample of RAW RGB images.
\medskip

%\secc Rectification of RGB images

%		\Green
%		\begitems \style o
%			* Popsat, jak opravím RGB snímek + obrázek
%		\enditems
%		\Black

\label[secc:bagSubRGB]
\secc Background Subtraction

In the chapter~\ref[sec:captureBackground] is described a way of capturing background image. Respectively was not captured a single image, but was captured a sequence of images. The sequence of images ${\bf B}^i_{R,G,B}(u,v)$ is averaged to ${\overline{\bf B}}_{R,G,B}(u,v)$~\ref[eq:averageRGB]. Together with mean value of RGB background is computed corrected sample standard deviation ${\bf \sigma}_{R,G,B}(u,v)$~\ref[eg:deviationRGB].
\label[eq:averageRGB]$$
{\overline{\bf B}}_{R,G,B}(u,v) = \sum_{i=1}^{N} {{{\bf B}^i_{R,G,B}(u,v)}\over{N}} \eqmark
$$
\label[eg:deviationRGB]$$
{\bf \sigma}_{R,G,B}(u,v)=\sqrt{{1\over N} \sum_{i=1}^{N} \left( {\bf B}^i_{R,G,B}(u,v) - {\overline{\bf B}}_{R,G,B}(u,v) \right)^2} \eqmark
$$
From sequence RGB images ${\bf I}^i_{R,G,B}(u,v)$ is computed a sequence of silhouettes ${\bf S}^i(u,v)$~\ref[eq:backgroundSubtraction].
\label[eq:backgroundSubtraction]
$$
{\bf S}^i(u,v) =
\cases{
1, &if $
\cases{
&$
\left| {\bf I}^i_{R}(u,v) - {\overline{\bf B}}_{R}(u,v)\right| >  {\bf \sigma}_{R}(u,v)
$;\cr
{\rm or} &$
\left| {\bf I}^i_{G}(u,v) - {\overline{\bf B}}_{G}(u,v)\right| >  {\bf \sigma}_{G}(u,v)
$;\cr
{\rm or} &$
\left| {\bf I}^i_{B}(u,v) - {\overline{\bf B}}_{B}(u,v)\right| >  {\bf \sigma}_{B}(u,v)
$;\cr
}
$\cr
0, &otherwise. 
\cr}
\eqmark
$$

$$
{\bf S}^i(u,v) =
\cases{
&$
\left| {\bf I}^i_{R}(u,v) - {\overline{\bf B}}_{R}(u,v)\right| >  {\bf \sigma}_{R}(u,v)
$;\cr
{\rm or} &$
\left| {\bf I}^i_{G}(u,v) - {\overline{\bf B}}_{G}(u,v)\right| >  {\bf \sigma}_{G}(u,v)
$;\cr
{\rm or} &$
\left| {\bf I}^i_{B}(u,v) - {\overline{\bf B}}_{B}(u,v)\right| >  {\bf \sigma}_{B}(u,v)
$;\cr
}
$$

		\Green
		\begitems \style o
			* Způsob filtrace proti pozadí - napsat vzorec
			* Popsat i použité morphologické operace pro zkvalitnění siluety (možná vlastní secc)
		\enditems
		\Black
Subtracted images are shown in the figure~\ref[fig:rgbsub].

\medskip \clabel[fig:rgbsub]{Background subtraction of RGB images}
\picw=18cm 
\cinspic pictures/rgbsub.pdf
\caption/f Sample of background subtraction of RGB images.
\medskip

%\secc Finding End of Gripper

%		\Green
%		\begitems \style o
%			* Popsat, jak naleznu oblast, kterou opisuje chapadlo při hýbání s látkou a jak z tohoto pohybu naleznu konec chapadlo v obraze
%			* Obrázek s vyznačenou kružnicí a bodem jako koncem gripperu
%		\enditems
%		\Black

\secc Finding Major Axis of the Garment and Remove Gripper

		\Green
		\begitems \style o
			* Napsat, jak hledám osu bramboroidu
			* Popsat zde zavrhnuté metody
			\begitems \style o
				* Kostra grafu + obrázky + proč jsem to nepoužil
				* Střed dle y osy + obrázek + proč jsem to nepoužil
			\enditems
			* Obrázky postupného nalezení osy bramborouidu (při aproximaci udělat více obrázků)
		\enditems
		\Black

Founded major and minor axes of the silhouette of the garment and subtraced images without the gripper are shown in the figure~\ref[fig:rgbsub].

\medskip \clabel[fig:rgbminoraxes]{Major and minor axes of the silhouette}
\picw=18cm 
\cinspic pictures/gripperMainAxisRGB.pdf
\caption/f Sample of major and minor axes of the silhouette of the garment and subtraced images without the gripper.
\medskip

\secc Finding Central Curve of the Garment

\secc Finding Mathematical Features from RGB images

		\Green
		\begitems \style o
			* Popsat, jak z osy bramboroidu naleznu body, které předávám jako výstup
			* Obrázky se siluetou a v ní s body
		\enditems
		\Black

