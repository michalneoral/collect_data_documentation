\label[sec:rgb]
\sec Extraction of Features from RGB images

The sequence of RGB images is captured with the xtion camera (chap.~\ref[secc:camera]). Images have resolution 640$\times$480~px (width$\times$height). RGB images are stored in the "bagfile"s like single row vector. After loading data into MATLAB~(chap.~\ref[sec:loadToMatlab]) data are converted from row representation to three-dimensional RGB intensity representation ${\bf I}_{u,v}$ of images with dimensions 480$\times$640~px ($m\times n$), where ${\bf I}_{u,v}=[~{r}(u,v),~{g}(u,v),~{b}(u,v)~]$, $u \in 1,2,\dots ,m$, $v \in 1,2,\dots,n$. ${r}(u,v)$, ${g}(u,v)$ and ${b}(u,v)$ are function of brightness of pixel on the coordinates $u,v$ and each of ${r}(u,v)$, ${g}(u,v)$ and ${b}(u,v)$ represents are relevant intensity of single color from the RGB color-space.
Unless specified otherwise, the computations are always carried for each ${r}(u,v)$, ${g}(u,v)$ and ${b}(u,v)$ separately.
Raw RGB data are shown in the figure~\ref[fig:extRGB].

\medskip \clabel[fig:extRGB]{Sample RAW RGB images}
\picw=18cm 
%\centerline {\inspic pictures/todo.pdf \hfil\hfil \inspic pictures/todo.pdf }\nobreak
%\centerline {a)\hfil\hfil b)}\nobreak\medskip
%\centerline {\inspic pictures/todo.pdf \hfil\hfil \inspic pictures/todo.pdf }\nobreak
%\centerline {c)\hfil\hfil d)}\nobreak\medskip
\cinspic pictures/rgbsample.pdf
\caption/f Sample of RAW RGB images.
\medskip

%\secc Rectification of RGB images

%		\Green
%		\begitems \style o
%			* Popsat, jak opravím RGB snímek + obrázek
%		\enditems
%		\Black

\label[secc:bagSubRGB]
\secc Background Subtraction

In the chapter~\ref[sec:captureBackground] is described a way of capturing background image. Respectively was not captured a single image, but was captured a sequence of $N$ images ${\bf F}^{i}=\{{\bf F}^{1},{\bf F}^{2},\dots,{\bf F}^{N}\}$ in the time $i \in \{1,2,\dots,N\}$, where ${\bf F}^i_{u,v}=[{r}(u,v),~{g}(u,v),~{b}(u,v)~]$.
The~sequence of background images ${\bf F}^i_{u,v}$ is averaged to ${\overline{\bf F}}_{u,v}=[{r}(u,v),~{g}(u,v),~{b}(u,v)]$ according to~\ref[eq:averageRGB]. Together with mean value ${\overline{\bf F}}_{u,v}$ of RGB background images, is computed corrected sample standard deviation ${\bf \sigma}_{u,v}=[~{\sigma_{R}}(u,v),~{ \sigma_{G}}(u,v),~{\sigma_{B}}(u,v)~]$~\ref[eg:deviationRGB], whith is used like intensity threshold.
\label[eq:averageRGB]$$
\forall u, v :  {\overline{\bf F}}_{u,v} = \sum_{i=1}^{N} {{{\bf F}^i_{u,v}}\over{N}} \eqmark
$$
\label[eg:deviationRGB]$$
\forall u, v :  {\bf \sigma}_{u,v}=\sqrt{{1\over {N-1}} \sum_{i=1}^{N} \left( {\bf F}^i_{u,v} - {\overline{\bf F}}_{u,v} \right)^2} \eqmark
$$

From the RGB image ${\bf I}$ is computed a binary silhouette image $s(u,v)$ according to~\ref[eq:backgroundSubtraction],where $\alpha > 1$ is a threshold coeficient.
\label[eq:backgroundSubtraction]
$$
\forall u, v :
{s}(u,v) =
\cases{
1, &if $
\cases{
&$
\left| {r}(u,v) - {\overline{\bf F}}_{R}(u,v)\right| >  \alpha \thinspace {\bf \sigma}_{R}(u,v)
$;\cr
{\rm or} &$
\left| {g}(u,v) - {\overline{\bf F}}_{G}(u,v)\right| >  \alpha \thinspace {\bf \sigma}_{G}(u,v)
$;\cr
{\rm or} &$
\left| {b}(u,v) - {\overline{\bf F}}_{B}(u,v)\right| >  \alpha \thinspace {\bf \sigma}_{B}(u,v)
$;\cr
}
$\cr
0, &otherwise. 
\cr}
\eqmark
$$

Silhouette images are shown in the figure~\ref[fig:rgbsub].

\medskip \clabel[fig:rgbsub]{Background subtraction of RGB images}
\picw=18cm 
\cinspic pictures/subtracked.pdf
\caption/f Sample of binary silhouette images - background subtraction.
\medskip

The figure~\ref[fig:rgbsub] shows that background subtraction is not perfect. This imperfection have two main reasons. First reason is that the the gripper of arm "r1" and the surrounding of monochrome background are partly seen in the image. This would not be a problem for our method, if these objects have not a shiny finish, which reflects the garment when the garment is moving. By experimental results, we found that the garment moves only against a monochrome background and not enter the background surroundings during the entire movement. This imperfection is removed with help of a clipping mask, which cover gripper and monochrome background surrounding.

Second reason is a noise on the RGB image and tiny changes of brightness of pictures, which is caused by the garment movement. This imperfection is removed with help of morphological operation {\bf opening}. Because this imperfection may occur in reverse too, ie so that the point of the garment is detected as a background thus {\bf closing} operation followed opening operation (mean of these morpholigical operation is explained in~\cite[hlavac2008] p.667-669). We use for this operations the MATLAB functions "imopen()" and "imclose()".

\medskip \clabel[fig:rgbsubmorph]{Background subtraction of RGB images after morphological operation}
\picw=18cm 
\cinspic pictures/rgbsub.pdf
\caption/f Background subtraction of RGB images after morphological operation.
\medskip


%\secc Finding End of Gripper

%		\Green
%		\begitems \style o
%			* Popsat, jak naleznu oblast, kterou opisuje chapadlo při hýbání s látkou a jak z tohoto pohybu naleznu konec chapadlo v obraze
%			* Obrázek s vyznačenou kružnicí a bodem jako koncem gripperu
%		\enditems
%		\Black

\secc Finding Major Axis of the Garment and Remove Gripper

%For further image processing, we must be sure that we have only one image object in the binary silhouette image. 
After applying the previous steps we can assume that the garment is represented by the largest connected region in the image. Therefore, we focuse on the largest region. We use MATLAB functions "bwconncomp ()", which finds all objects in an image. For these object is individually calculated their size.  Subsequently, according to these sizes determine the largest object and take out the rest. The detailed explanation of these region identification is described in the~\cite[hlavac2008]~(p.332).

In the next step we found major and minor axis of the garment. Befere we can find these parameters we have to remove from the image a showed end of the gripper. We have captured images without measured garment in the gripper, thus we can find an area in which the gripper is. Then this part of the image is removed from the image of the measured garment. A better way would be to compare the captured image data from a simulated robot model in an environment of ROS.

After the garment was removed from the gripper, the major and minor axis could be found by using the spatial moments (described in~\cite[haralock1991computer] p.74,658). This method describes finding major and minor axis for an ellipse (reasons are described below in the chap.~\ref[secc:centralCurve]). The axes are found using MATLAB functions "regionprops()" which gave us center of the object, lenght of major and minor axis and clockwise orientation of the major axis.
Founded major and minor axes of the silhouette of the garment and subtraced images without the gripper are shown in the figure~\ref[fig:rgbsub].

\medskip \clabel[fig:rgbminoraxes]{Major and minor axes of the silhouette}
\picw=18cm 
\cinspic pictures/gripperMainAxisRGB.pdf
\caption/f Sample of major and minor axes of the silhouette of the garment and subtraced images without the gripper.
\medskip

\label[secc:centralCurve]
\secc Finding Central Curve of the Garment

To obtain the garment positions, we decided to track a central curve of the silhouette. During this project we have tried various methods of finding the curve, for example, the average pixel with respect to the rows of image region, or find a skeleton of the silhouette image (\cite[hlavac2008],~p.365). Search skeleton of the silhouette image it seems like the most promising from the beginning, but finally there was a problem in the form of determining the curve at the ends of the garment.

Finally, we chose finding of the central curve so that the search for the centers of the corresponding cuts, which are perpendicular to the major axis of the silhouettes.

\Blue
The first step to obtaining the central curve of the silhouette is finding the the outline of the silhouette. I used the another property of the MATLAB function "regionprops()", which contains a edge pixels of the garment. Edge of the silhouette is shown in the figure~\ref[fig:rgbedge]. For the next processing is important to convert pixels coordinates of the edge of the garment to the border curve. For this conversion we use inner boundary tracing algorithm (which is described in~the~\cite[hlavac2008] p.191). In our case, the MATLAB function "bwboundaries()" is used as the implementation of this algoritm.
\Black

\midinsert \clabel[fig:rgbedge]{Edge of the silhouette}
\picw=18cm 
\cinspic pictures/silhouette.pdf
\caption/f Sample of edge of the silhouette of the garment.
\endinsert

\Blue
The final step is to find the central curve. The point of the central curve is founded with help of the corresponding cut line as is described earlier in this chapter. The center of the silhouette is represented like midpoint of the point, which are compuded as intersections of the border curve and corresponding cut line. For computing this curve intersections is used the fast and robust curve intersections algorithm~\cite[curveIntersect].
Founded center curve of the silhouette is shown in the figure~\ref[fig:rgbedge].
\Black



\secc Finding Mathematical Features from RGB images

To estimate the parameters of the garment, we wanted to use the points of the central curve of the silhouette over time.
Unfortunately, the position of the points forming the central curve is not only influenced by observed movement of the garment, initialized by the robot.
The silhouette and the central curve is influenced by the complex \mind{CHYBÍ SLOVO} of the garment. It is not suitable for basic experiments. We decided not to use RGB images to estimate the parameters of dynamic physical model of the garment.

\medskip \clabel[fig:centerSil]{Founded center curve of the silhouette}
\picw=18cm 
\cinspic pictures/rgbcentercurve.pdf
%\cinspic pictures/rgbsample.pdf
\caption/f Founded center curve of the silhouette.
\medskip

\bigskip
\bigskip
\bigskip
\bigskip
\bigskip
\bigskip
\bigskip
\bigskip
\bigskip
