% for using this tex sources you must have ctustyle.tex template
% by RNDr. PETR OLŠÁK
% http://petr.olsak.net/ctustyle.html


\input ctustyle

\def\thednum{(\the\chapnum.\the\dnum)}

\def\mind#1{\Red{\bf#1}\Black}
\def\dopln{\mind{DOPLŇ~ODKAZ!}}
\def\task#1{\Green{#1}\Black}
\def\lorem{\mind{Lorem ipsum dolor sit amet, consectetur adipiscing elit. Curabitur arcu dolor, congue sit amet pellentesque vitae, semper ac ligula. Nam eu elit erat. Phasellus pretium at tellus sed commodo. Quisque auctor metus eu diam pretium pulvinar. Curabitur posuere ligula sed dapibus aliquet. Sed aliquet malesuada felis, vel aliquam lacus laoreet non. Pellentesque in tristique erat. Ut mollis consequat interdum. Integer orci velit, venenatis sed ligula et, pretium dapibus sem. Nam rhoncus urna sed urna volutpat congue. Ut suscipit ipsum vel.}}
%\def\lorem{\mind{Vymyslet text}}
\def\newbeg{}
\def\newend{}


\worktype [B/EN]
\faculty    {F3} 
\department {Department of Cybernetics}
\title      {Extraction of Features\nl from Moving Garment}
%\subtitle   {}
\author     {Michal Neoral}
\date       {Květen 2014}
\supervisor {Ing. Pavel Krsek, Ph.D.}
\studyinfo  {CYBERNETICS AND ROBOTICS, Robotics}  % Study programme etc.

%\workinfo   {}
\titleCZ    {Získání příznaků z obrazu pohybující se látky}
%\subtitleCZ {}
\pagetwo    {}  % The text printed on the page 2 at the bottom.

\specification {\picw=\hsize \cinspic pdf/specAJ.png
				\vfil\break \cinspic pdf/specCZ.png }

\abstractEN {
   This... 
This thesis has goal to design method of measurement and extraction of image features for obtaining parameters which will lead to construct the model.
		\mind{do 11.5.}
}
\abstractCZ {
   Tento... 

		\mind{do 11.5.}
}           % If your language is Slovak use \abstractSK instead \abstractCZ

\keywordsEN {%
   dynamic model; garment model; feature extraction; 3D~image; silhouette.
}
\keywordsCZ {%
   dynamický model; model oděvu, textilie; extrakce příznaků; 3D~obraz; silueta.
}
\thanks {           % Use main language here
   Foremost, I would like to thank to Ing. Pavel Krsek, Ph.D. ...
		
   \mind{do 11.5.}
}
\declaration {      % Use main language here
   Prohlašuji, že jsem předloženou práci vypracoval
   samostatně a že jsem uvedl veškeré použité informační zdroje v~souladu
   s~Metodickým pokynem o~dodržování etických principů při přípravě
   vysokoškolských závěrečných prací.

   V Praze dne 5. 5. 2013 % !!! Attention, you have to change this item.
   \signature % makes dots
}

%%%%% <--   % The place for your own macros is here.

%\draft     % Uncomment this if the version of your document is working only.
%\linespacing=1.7  % uncomment this if you need more spaces between lines
                   % Warning: this works only when \draft is activated!
%\savetoner        % Turns off the lightBlue backround of tables and
                   % verbatims, only for \draft version.
%\blackwhite       % Use this if you need really Black+White thesis.
%\onesideprinting  % Use this if you really don't use duplex printing. 

\makefront  % Mandatory command. Makes title page, acknowledgment, contents etc.

% Úvod
\input introduction

% Kapitoly
\input description
\input datagetting
\input fileformat
\input dataprocessing
\input result
\input discussion

% Závěr
\input conclusion

% Bibliografie
\bibchap
\usebbl/c ref  % finalní verze musí být /c

% Přílohy
%\input specifications
\input dvdContent
\input specialcases
%\input otherImages
\input shortcuts
\input manualcollect
%\input automaticcollect

\bye
