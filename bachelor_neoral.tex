% for using this tex sources you must have ctustyle.tex template
% by RNDr. PETR OLŠÁK
% http://petr.olsak.net/ctustyle.html


\input ctustyle

\def\thednum{(\the\chapnum.\the\dnum)}

\def\mind#1{\Red{\bf#1}\Black}
\def\dopln{\mind{DOPLŇ~ODKAZ!}}
\def\task#1{\Green{#1}\Black}
\def\lorem{\mind{Lorem ipsum dolor sit amet, consectetur adipiscing elit. Curabitur arcu dolor, congue sit amet pellentesque vitae, semper ac ligula. Nam eu elit erat. Phasellus pretium at tellus sed commodo. Quisque auctor metus eu diam pretium pulvinar. Curabitur posuere ligula sed dapibus aliquet. Sed aliquet malesuada felis, vel aliquam lacus laoreet non. Pellentesque in tristique erat. Ut mollis consequat interdum. Integer orci velit, venenatis sed ligula et, pretium dapibus sem. Nam rhoncus urna sed urna volutpat congue. Ut suscipit ipsum vel.}}
%\def\lorem{\mind{Vymyslet text}}
\def\newbeg{}
\def\newend{}


\worktype [B/EN]
\faculty    {F3} 
\department {Department of Cybernetics}
\title      {Extraction of Features\nl from Moving Garment}
%\subtitle   {}
\author     {Michal Neoral}
\date       {May 2014}
\supervisor {Ing. Pavel Krsek, Ph.D.}
\studyinfo  {CYBERNETICS AND ROBOTICS, Robotics}  % Study programme etc.

%\workinfo   {}
\titleCZ    {Získání příznaků z obrazu pohybující se látky}
%\subtitleCZ {}
\pagetwo    {}  % The text printed on the page 2 at the bottom.

\specification {\picw=\hsize \cinspic pdf/AJ.pdf
				\nextoddpage
				\vfil\break \cinspic pdf/CZ.pdf }

\abstractEN {
The point of this work is to~design a~method to obtain image features of~the~moving garment. This features should be further used to estimate the parameters of the dynamic model of the garment.

For the purpose of obtaining the parameters of the dynamic model, it is observed the dynamic behavior of the real garment. This dynamic behavior is caused by a short and simple movement of a robotic manipulator which holds the garment in the gripper.

I concentrated mainly on the obtaining these parameters from time series of data from the rangefinder sensor and from a standard color camera, which is the movement of the garment captured.

Designed solution could be, combined with a model of the garment, provide better  possibilities for motion simulation, motion planning and collision detection in robotic manipulation with the garments.

%Basis of our measurements is the movement of fabric, or garment, in front of the sensor using a robotic manipulator, which was controled by robot operating system, ROS.
%The image data were processed using mathematical programming environment MATLAB.
As part of this work was captured image data consisting of forty-six measurements for eight different types of the garment.
The method of computation was verified on that data set.

The experiments verified that the proposed method has satisfactory repeatability.
%The performed experiments we found that the proposed method has high repeatability, becouse it was confirmed that the same garment in the same motion behaves approximately as same way.
%Based on the parameters should be possible to create a virtual model of dynamic physical model of the garments.
%Part of this work is the data set of various garments that can serve as a set for the validation of estimated parameters.
%This thesis has goal to design method of measurement and extraction of image features for obtaining parameters which will lead to construct the model.
}
\abstractCZ {
Cílem této práce je navržení metody pro získání obrazových příznaků z pohybující se látky. Tyto příznaky by měly být dále využity pro získání parametrů dynamického modelu látky. %, která by byla využitelná v rámci mezinárodního projektu CloPeMa, jenž se specializuje na robotickou manipulaci s oděvy.

Pro účely získání parametrů dynamického modelu sledujeme dynamické chování reálné látky. Toto dynamické chování je vyvoláno krátkým a jednoduchým pohybem robotického manipulátoru, který drží tuto látku v chapadle.

%Podstatou našeho měření je pohyb látky, či oděvu, před senzorem pomocí robotického manipulátoru, který byl ovládán pomocí robotického operačního systému ROS.
Zaměřil jsem se převážně na získání těchto parametrů z časové posloupnosti dat z hloubkového senzoru a standardní barevné kamery, jimiž je pohyb látky snímán.

%Obrazová data jsem zpracovával pomocí matematického programovacího prostředí MATLAB. 
Vytvořené řešení by mohlo v kombinaci s modelem látky poskytnout lepší možnosti simulace pohybu, plánování pohybu a detekci kolizí při robotické manipulaci s látkami. 
 
%Provedeným výzkumem jsme zjistili, že navrhnutá metoda má velkou opakovatelnost, jelikož bylo potvrzeno, že se stejná látka při stejném pohybu chová přibližně stejně.
%Na základě zjištěných parametrů by mělo být možné sestavit virtuální model dynamického fyzikálního modelu látky.
V rámci této práce byla pořízena obrazová data sestávající z čtyřiceti šesti měření pro osm různých látek. 
Metoda výpočtů byla ověřována na těchto datech.

Provedené experimenty ověřily, že navržená metoda má dostatečnou opakovatelnost.
}

\keywordsEN {%
   dynamic model; garment model; feature extraction; 3D~image; silhouette.
}
\keywordsCZ {%
   dynamický model; model oděvu, textilie; extrakce příznaků; 3D~obraz; silueta.
}
\thanks {           % Use main language here
   Foremost, I would like to thank to Ing. Pavel Krsek, Ph.D. for his great guidance and help throughout this project.
I would also like to thank Ing. Libor Wagner and Bc. Vladimír Petrík for their help with the robot experiments.
Finally, I would like to thank my parents for their support during my studies.
}
\declaration {      % Use main language here
   Prohlašuji, že jsem předloženou práci vypracoval
   samostatně a že jsem uvedl veškeré použité informační zdroje v~souladu
   s~Metodickým pokynem o~dodržování etických principů při přípravě
   vysokoškolských závěrečných prací.

   V Praze dne 5. 5. 2013 % !!! Attention, you have to change this item.
   \signature % makes dots
}

%%%%% <--   % The place for your own macros is here.

%\draft     % Uncomment this if the version of your document is working only.
%\linespacing=1.7  % uncomment this if you need more spaces between lines
                   % Warning: this works only when \draft is activated!
%\savetoner        % Turns off the lightBlue backround of tables and
                   % verbatims, only for \draft version.
%\blackwhite       % Use this if you need really Black+White thesis.
%\onesideprinting  % Use this if you really don't use duplex printing. 

\makefront  % Mandatory command. Makes title page, acknowledgment, contents etc.

% Úvod
\input introduction

% Kapitoly
\input description
\input datagetting
\input fileformat
\input dataprocessing
\input result
%\input discussion

% Závěr
\input conclusion

% Bibliografie
\bibchap
\usebbl/c ref  % finalní verze musí být /c

% Přílohy
%\input specifications
\input dvdContent
\input descdata
\input specialcases
%\input otherImages
\input shortcuts
%\input manualcollect
%\input automaticcollect

\bye
