
\sec Extraction of Features from Depth Map


The sequence of depth maps is captured with the Xtion camera (chap.~\ref[secc:camera]). Depth maps are in a sufficient resolution 640$\times$480~px (width$\times$height). Depth maps are stored in the "bagfile"s like single row vector. After loading the data into MATLAB~(chap.~\ref[sec:loadToMatlab]) data are converted from row representation in milimeters to depth map ${\rm C}(u,v)=z$, where $z$~is a distance from the Xtion camera in meters, $u$ and $v$ are the depth map image coordinates ($ u \in 1, 2, \dots, n; v \in 1, 2, \dots, m $). The depth maps are shown figure~\ref[fig:extDepth]. 

\medskip \clabel[fig:extDepth]{Illustration Depth Map from Rangefinder}
\picw=18cm
\cinspic pictures/depth.pdf
\caption/f Depth Map obtained by Xtion rangefinder.
\medskip

\secc Convert Depth Map to 3D points


Together with data depth map is also stored calibration parameters of the Xtion rangefinder. From these calibration parameters comes calibration matrix ${\bf K}$~(eq.~\ref[eq:calib]), where $f_x$ and $f_y$ are partial focal lengths and $(c_x, c_y)$ is a principal point.  
\label[eq:calib]
$${\bf K} = \left[ \matrix{ 
f_x & 0 & c_x \cr
0 & f_y & c_y\cr
0 & 0 & 1 \cr}
\right] \eqmark $$
Using the calibration matrix ${\bf K}$ can be depth map ${\rm C}(u,v)$ converted into 3D point cloud (set of points) ${\bf M}$~(eq.~\ref[eq:muv]), where ${\bf M}_i = [ x_i, y_i, z_i]$ is point represent by its euclidean coordinates. We use usually coordiate system of the Xtion sensor.
\label[eq:muv] 
$${\bf M} = \{ {\bf M}_1, {\bf M}_2, \dots, {\bf M}_{m\times n} \} \eqmark$$
The point cloud ${\bf M}$ is constructed according to equation~\ref[eq:get3Dpoints]. Algoritmus used for these convertion was inspirated by~\cite[matlabpointcloud]. We obtain each point ${\bf M}_i \in {\bf M}$ from depth map ${\rm C}(u,v)$ by:
\label[eq:get3Dpoints]
$$
\eqalignno{
z_i &= {\rm C}(u,v)\cr
x_i &= {z \cdot (u-c_x) \over {f_x}} & \eqmark \cr
y_i &= {z \cdot (v-c_y) \over {f_y}} \cr
}
$$
where $u \in 1, 2, \dots, n$, $v \in 1, 2, \dots, m$ and $i \in 1, 2, \dots, n\times m$.
Example of data from 3D points are shown in the figure~\ref[fig:pointsDepth].

\medskip \clabel[fig:pointsDepth]{Examples of point cloud of a garment}
\picw=18cm
\cinspic pictures/point3D.png
\caption/f Examples of point cloud of a garment.
\medskip



\label[secc:findEE]
\secc Finding End of the Gripper

In the figure~\ref[fig:findPoints] are not shown only points on the garment. The figure shows points of the gripper of arm "r2"~\ref[secc:camera] too. 
As has been said in chapter~\ref[sec:capDepth] the movement of the garment is caused by this gripper. Finding the coordinates ${\bf ee}~=~[ee_x,ee_y,ee_z]$ of the end point of the gripper is important not only for resoluting the gripper from the garment, but also for detecting excitation movement of garment. 

The coordinates are found by information about robot position. The simple service in ROS was written for this operation~(chap.~\ref[secc:rosintro]). 
This service subscribes to the topic of "/xtion1/depth/image_raw". This topic generates a message each time an depth map image is captured. For every incoming message is compuded a coordinates of ${\bf ee}$ --- end of the gripper with garment.
To obtain these coordinates is used "lookupTransform()" function from the ROS environment. The function uses the "tf" package of ROS which contains the relationships between parts of manipulator in a tree structure and allows to obtain a transformation between any two portions of the manipulator~\ref[rostf]. 
This coordinates are not in the global coordinate system, but in the coordiate system of "xtion1" camera. Then, the service publishes data as a new message in its own topic. So the position of the end point of the gripper is captured for every depth map image. Founded gripper is shown in the figure~\ref[fig:find3DGripper].

%\label[eq:endEffector]
%$$ ee_{x,y,z} \eqmark $$


\medskip \clabel[fig:find3DGripper]{3D filtred points with founded gripper}
\picw=18cm
\cinspic pictures/point3Dgripper.pdf
\caption/f Example of 3D filtred points with founded gripper (gripper position shows a red circle). 
\medskip   


\secc Filtering by Depth of Area

The figure~\ref[fig:pointsDepth] shows that the Xtion camera captured not only the points on the garment but also captured the unnecessary surroundings. We decided filtering the 3D points according to distance from Xtion camera. We know that the~gripper of the arm "r2" with garment is located 1.06--1.07~m
 before the Xtion camera on the arm "r1" at its default measurement position. After several experiments, we found that the minimum and maximum of z-axis values on the garment have values between 0.8~m to 1.16~m, thus we know that the garment moves between these values. 
%Maximum deviation $\Delta z$ from the z-axis value of endpoint $ee_z$ is computed by equation \ref[eq:minmax]. 
%\label[eq:minmax] 
%$$\Delta z=max(abs(ee_z-min_{garment}(z)),abs(ee_z-max_{garment}(z))) \eqmark$$

If the deviation is known, point cloud ${\bf M}$~(from~\ref[eq:get3Dpoints]) can be filtered to point cloud ${\bf M}^*$~(eq.~\ref[eq:mstar]), to extract the garment shape~(according to eq.~\ref[eq:filter3D]).
\label[eq:mstar]
$${\bf M}^* = [ x^*, y^*, z^* ] \eqmark$$
The filted data are shown in the figure~\ref[fig:pointsFiltering]. You can see the differences between figures~\ref[fig:extDepth] and~\ref[fig:pointsFiltering]. Moreover, this method of ''background substraction'' is much robust than method of background substraction of RGB images~\ref[secc:bagSubRGB].
\label[eq:filter3D]
%$${\bf M}_j \subseteq {\bf M}_{i} \in ee_z - 3\cdot \Delta z \leq z_i \leq ee_z + 3\cdot \Delta z \eqmark$$ 
$$
\eqalignno{
{\bf M}^* &\subseteq {\bf M}\cr
{\bf M}^* = \{ {\bf M}_i = [x_i,y_i,z_i];~ & ee_z - \Delta z \leq z_i \leq ee_z + \Delta z \} & \eqmark \cr
}
$$
when $\Delta z$ is range of motion of the garment according to the gripper coordinates. We choose a~value of $\Delta z$ three times to value determine on the base of experimental results. 


\medskip \clabel[fig:pointsFiltering]{Filtred point cloud of a garment}
\picw=18cm
\cinspic pictures/point3Dfiltred.pdf
\caption/f Example of filtred point cloud of a garment.
\medskip

\label[secc:findingPointsDepth]
\secc Finding Points

First of all is reduced the 3D points matrix. Points over the gripper endpoint ${\bf ee}$~(from chap.~\ref[secc:findEE]) are not needed anymore. 
%This points are removed from the set ${\bf M}_{j}$~(eq.~\ref[eq:reducedFromEE]).
%\label[eq:reducedFromEE] 
%$$
%{\bf M}_k   {\bf M}_{j} \in  y_i < {\bf ee}_y
%\eqmark$$ 
%----------------------------------------------------------------------
%----------------------------------------------------------------------
%----------------------------------------------------------------------
%----------------------------------------------------------------------
The points are filtred by the y-axis from point cloud ${\bf M}^*$ to the point cloud ${\bf G}$.

We suppose a simple model usefull for parametric estimation. The model can be imagined as a chain of connected segments pendulous vertically down. We need to obtain the position of points on the garment vertically down from the gripper. 
%To preserve the Shannon sampling theorem, must be the number of samples at least twice greater than the number of period function. Because we assume that this function consists of one or two periods of the sinusoidal waveforms, the number of these points will sufficient. 
Vertical direction will be denoted as the direction parallel to the y axis. 

When the movement is limited, we can assume that the point cloud ${\bf G}$ represents a some function $z={\rm f}(y)$, where $y$ is the vertical coordiate of the point on the garment and for all $z$ will be the $x^i$ (at a given $y^i$) approximately constant. This function can be obtained from the point cloud by sampling. 
%Then it is necessary to sample the garment.
Sampling is chosen in the vertical direction down from gripper position. 
Sampling intervals are chosen based on a raster which provides a sensor. The raster of distance is approximately (on our dataset) $\Delta y_{us} = 2.5~mm$. So we choose the sampling interval $\Delta y = 5~mm$, to we almost always received at least one point. We also it is set as the sampling size $\Delta x$ between the points and y-axis the gripper, which is parallel to the y-axis. 
%For simplicity, we will from now denote the 3D point $[x, y, z]$ as ${\bf v} = [x,y,z]$, so ${\bf G}_i=[x_i,y_i,z_i]$ is the same like ${\bf G}_i={\bf v}_i$.
We found sets ${\bf L^i}$, where $i \in \{1,2,\dots,k\}$ and $k$ is number of sampling intervals vertically down from the gripper to end of the garment. Let ${\bf L}^i = \{ {\bf p}^i_1, {\bf p}^i_2, \dots, {\bf p}^i_o \}$, where $o$ is number of points contains in ${\bf L}_i$, than the points ${\bf p}^i_j$ (${\bf p}^i_j \in {\bf G}$) meet the following conditions:
%$$
%{\bf L}_i = \{ \forall [x_j,y_j,z_j] \subseteq {\bf G} ~|~ j \in 1,2,\dots,n ~| (x-\Delta x) < s_x < (x + \Delta x) ~|~ (y-\Delta y) < s_y < (y + \Delta y) \} 
%}\eqmark
%$$
$$
\eqalignno{
{\bf p}^i_j &= \left[x^i_j,y^i_j,z^i_j\right] \cr
\forall i,j : (x^i- &\Delta x) < x^i_j < (x^i + \Delta x) \cr
\forall i,j : (y^i- &\Delta y) < y^i_j < (y^i + \Delta y) & \eqmark \cr
}
$$
Than ${\bf s}^i = [x^i,y^i,z^i]$ belongs to ${\bf L}_i$, where we know $x^i$ and $y^i$, is determined like:
$$
{\bf s}^i = \left[ x^i, y^i, z^i = {1\over o} \sum_{j=1}^o z^i_j \right] \eqmark
$$
where $o$ is the number of elements of ${\bf L}^i$. This we found the chain of the sample points ${\bf s}^i$, where ${\bf s}^i$ is sorted according to $y^i$.
%All these averaged points ${\bf s} = [x, y, z ]$ whith includes a set of points determined using eq.~\ref[eq:sampling] are averaged to ${\bf S}_i=[s_x, s_y, s_z,]$. Herewith it is also ensured filtering of points.  
%\label[eq:sampling]
%$$
%{\bf S} = \{ {\bf S}_i \in {\bf G}; (x-\Delta x) < s_x < (x + \Delta x); (y-\Delta y) < s_y < (y + \Delta y) \} \eqmark  
%$$
%$$
%\eqalignno{
%\forall {\bf s} : (x-\Delta x) < &s_x < (x + \Delta x)\cr
%(y-\Delta y) < &s_y < (y + \Delta y) & \eqmark \cr 
%}
%$$

In the thesis~\cite[pecha2014bac] was made the requirement for location of measurement points ${\bf P}=\{ {\bf P}_1, {\bf P}_2, \dots, {\bf P}_r\}$ on the garment, where ${\bf P}_q=[x_q,y_q,z_q]$ and $r$ is number of required points.
According to the require points are located in the middle of equally long segments and value of require points $r$ is five (but it is changeable).
Therefore, the next step is finding points based on constant length of the garment.

First step is found lenght of the garment. The $l$ length of garment is sumarized according to sample points ${\bf s}_i$~(eq.~\ref[eq:sum]).
\label[eq:sum]
$$
\forall {\bf s}_i : l = \sum_{i=1}^{j-1} || {\bf s}_{i} - {\bf s}_{i+1}||  \eqmark
$$
%$$
%\forall {\bf s}_i : l = \sum_{i=1}^{n-1} \sqrt{ ( {\bf S}_{iy} - {\bf S}_{(i+1)y} )^2 +  ( {\bf S}_{iz} - {\bf S}_{(i+1)z} )^2} \eqmark
%$$
where $j$ is the number of elements of ${\bf s}$.
The length $l$ of the garment is divided by the number of required points. With this dividing are made equally long segments.
%Searched points are found at a constant distance given length of the garment. 
The required points ${\bf P}$ are in the middle of these segments. 
These points (with time stamps) are forwarded for further processing. 
Founded points are shown in the figure~\ref[fig:findPoints].

\medskip \clabel[fig:findPoints]{3D point cloud with founded points}
%\picheight=15cm 
\picw=18cm
%\centerline {\inspic pictures/todo.pdf \hfil\hfil \inspic pictures/todo.pdf }\nobreak
%\centerline {a)\hfil\hfil b)}\nobreak\medskip
%\centerline {\inspic pictures/todo.pdf \hfil\hfil \inspic pictures/todo.pdf }\nobreak
%\centerline {c)\hfil\hfil d)}\nobreak\medskip
\cinspic pictures/fig.pdf
\caption/f Example of 3D point cloud with points found on the garment.
\medskip

\bigskip
\bigskip
\bigskip


\secc Finding Mathematical Features from Depth Map

Points ${\bf P}$ (from chap.~\ref[secc:findingPointsDepth]) are obtained from whole sequence of the measurement of the garment. Because we found points P including timestamps, we can examine the dependence of displacement on time or dependence of speed on time. The timestamps for each depth map we obtained also from rosbag file.

The dependence of displacement $\Delta z_k$ on the time is shown in the figure~\ref[fig:graphPosition] and dependence of speed of the points on the time is shown in the figure~\ref[fig:graphSpeed]. Figures show founded points ${\bf P}$ (blue) and also they show position or speed of the end of the gripper ${\bf ee}$. Both figures show specified dependences to green towel captured in the special case of holding of the garment straightened on the ruler. \mind{Přidat ještě jeden případ do přílohy?}

\eject

\medskip \clabel[fig:graphPosition]{Dependence of the position of points on the time}
\picw=18cm 
\cinspic eps/graph5_3_5.pdf
\caption/f Example of the movement of the garment, dependence of the position of points on the time.
\medskip

\medskip \clabel[fig:graphSpeed]{Dependence of the speed of points on the time}
\picw=18cm 
\cinspic eps/graphSpeed.pdf
\caption/f Example of the movement of the garment, dependence of the speed of points on the time.
\medskip
