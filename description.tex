\chap Description of Workplace and of the Software

\label[sec:workplace]
\sec Workplace

\midinsert
\picw=8cm \cinspic pictures/robotCTU-eps-converted-to.pdf
\caption/f Manipulator of CloPeMa project location at CTU
\endinsert

\secc Manipulator

The base is~composed of~two robotic manipulator arm Motoman~MA1400. First arm is~called as~"r1" (or~also~appears like "R1"). Second arm is similarly marked "r2"~("R2"). The arms "r1" and "r2" are placed on~the~turntable. The turntable is rotated about an~axis known as~"external axis" (or~"Ext." or~possibly as~axis~"13"). Location of~arms and rotating around the~"Ext."~axis can be~better seen from (figure~\ref[fig:motomanAndTable]). Each arm of~manipulator has 6~axes, which is~able to~rotate. The~axes are labeled according to the manufacturer with the~letters "S", "L", "U", "R", "T" and~"B"~(figure~\ref[fig:motomanAxis]). This~designation is~not~enough to~for~recognize them and name of~letters was assigned the~number of~the~on~which this axis are located. Eg.: "S"~axis located on~the~arm~"r1" will be~called "S1", etc. Similarly to~the~designation of~arms we can meet even using small letters~(eg.:~"s1").

\label[fig:motomanAndTable]
\medskip
\picw=8cm \cinspic pictures/motomanCelek-eps-converted-to.pdf
\caption/f Identification of arms and location of external axis.
\medskip

\label[fig:motomanAxis]
\medskip
\picw=8cm \cinspic pictures/motomanMA1400axis-eps-converted-to.pdf
\caption/f Description of robotic arm Motoman~MA1400 - axis.
\medskip

\secc End effector

Each of~arms "r1" and "r2" are ended with eletricly controlled grippers. (figure~\ref[fig:gripper]). Grippers are used to~grasp of~garment. 

\label[fig:gripper]
\medskip
\picw=8cm \cinspic pictures/gripperXtion-eps-converted-to.pdf
\caption/f End effector (gripper). a)~gripper, b)~sensor Asus Xtion, c)~end of~arm on~which the~gripper is mounted.
\medskip

\label[secc:camera]
\secc Sensors %Camcorder

The~next important part of~manipulator is~camcorder Asus Xtion. This camcorder is~able~to~record RGB~images and depth maps. Camcorder mounted on~the~arm~"r1" is~called "xtion1" and camcorder mounted on~the~arm~"r2" is~called "xtion2". Position of~cameras is~shown in~figure~\ref[fig:gripper].



\sec Software

Robot is operated using Robot Operating System (ROS). ROS is an open-source system. ROS is not an operating system in the traditional sense of process management and scheduling. Rather, it provides a structured communications layer above the host operating systems of a heterogenous compute cluster~\cite[quigley2009ros]. In CloPeMa project is used Ubuntu as a host operating system.
\mind{Psát i hardware? Zmínit více i použití UBUNTU?}


\label[secc:rosintro]
\secc Brief Introduction to the Robot Operating System 

A system built using ROS consists of a number of processes, potentially on a number of different hosts, connected at runtime in a P2P topology. The fundamental concepts of the ROS implementation are {\bf nodes}, {\bf messages}, {\bf topics}, and {\bf services}. 

{\bf Nodes} are processes that perform computation. ROS is designed to be modular. A system is typically comprised of many nodes. In this context, the term ''node'' is interchangable with ''software module''. Nodes communicate with each other by passing messages. A {\bf message} is a a strictly typed data structure. Standard primitive types (integer, floating point, boolean, etc.) are supported. Arrays of primitive types and constants are supported too. Messages can be composed of other messages, and arrays of other messages, nested arbitrarily deep. A node sends a message by publishing it to a given {\bf topic}. A node that is interested in a certain kind of data will subscribe to the appropriate topic. There may be multiple concurrent publishers and subscribers for a single topic, and a single node may publish and/or subscribe to multiple topics. In general, publishers and subscribers are not aware of each others existence~\cite[quigley2009ros].

Although the topic-based publish-subscribe model is a flexible communications paradigm, its ''roadcast'' routing scheme is not appropriate for synchronous transactions, which can simplify the design of some nodes. In ROS, we call this a {\bf service}, defined by a string name and a pair of strictly typed messages: one for the request and one for the response. This is analogous to web services, which are defined by URIs and have request and response documents of well-defined types. Note that, unlike topics, only one node can advertise a service of any particular name: there can only be one service called ''classify imag'', for example, just as there can only be one web service at any given URI~\cite[quigley2009ros].
\mind{Je podstatný tento odstavec? Nikde o službách nehovořím.}

In the ROS are designed a large number of tools e.g. for get and set configuration parameters, for plotting or visualisation. For this project is important a {\bf rosbag tool}. This is basucally a set of tools for recording from and playing back to ROS topics~\cite[rosbag]. With help of this tool we can record choosen topics, including timestamp, to the "*.bag" file.


