\label[chap:dataproc]
\chap Data Processing

		\Green
		\begitems \style o
			* U všech obrázku níže udělat alternativu (lépe dvě) a tu vložit do přílohy a odkázat na ní
		\enditems
		\Black

\sec Load Data to the MATLAB

		\Green
		\begitems \style o
			* Popis + použitý kód
		\enditems
		\Black

\sec Extraction of Features from RGB

		\Green
		\begitems \style o
			* Pohovořit v kostce, co očekávám, že dostanu z těchto dat a stručně, jak budu postupovat.
			* Obrázek syrových dat 4 pcs
		\enditems
		\Black

\secc Rectification of RGB

		\Green
		\begitems \style o
			* Popsat, jak opravím RGB snímek + obrázek
		\enditems
		\Black

\secc Filtering Background by Use Reference Image

		\Green
		\begitems \style o
			* Způsob filtrace proti pozadí - napsat vzorec
			* Popsat i použité morphologické operace pro zkvalitnění siluety (možná vlastní secc)
			* Obrázek siluety 4 pcs
		\enditems
		\Black

\secc Finding End of Gripper

		\Green
		\begitems \style o
			* Popsat, jak naleznu oblast, kterou opisuje chapadlo při hýbání s látkou a jak z tohoto pohybu naleznu konec chapadlo v obraze
			* Obrázek s vyznačenou kružnicí a bodem jako koncem gripperu
		\enditems
		\Black

\secc Finding Central Curve of Garment

		\Green
		\begitems \style o
			* Napsat, jak hledám osu bramboroidu
			* Popsat zde zavrhnuté metody
			\begitems \style o
				* Kostra grafu + obrázky + proč jsem to nepoužil
				* Střed dle y osy + obrázek + proč jsem to nepoužil
			\enditems
			* Obrázky postupného nalezení osy bramborouidu (při aproximaci udělat více obrázků)
		\enditems
		\Black

\secc Finding Mathematical Features from RGB

		\Green
		\begitems \style o
			* Popsat, jak z osy bramboroidu naleznu body, které předávám jako výstup
			* Obrázky se siluetou a v ní s body
		\enditems
		\Black






\sec Extraction of Features from Depth Map

		\Green
		\begitems \style o
			* Pohovořit v kostce, co očekávám, že dostanu z těchto dat a stručně, jak budu postupovat.
			* Obrázky surových dat - depthmap
		\enditems
		\Black

\secc Rectification of Depth Map to 3D points

		\Green
		\begitems \style o
			* Popsat způsob, napsat vzorec
			* obrázky předělaného depthmap do 3D points
		\enditems
		\Black

\secc Filtering by Depth of Area

		\Green
		\begitems \style o
			* Jak filtruji dle vzdálenosti pouze tak, aby mi zůstala hýbající se látka
			* Obrázek filtrovaný dle hloubky
		\enditems
		\Black

\secc Finding Points

		\Green
		\begitems \style o
			* Nalezení gripperu
			* Vytvoření pole bodů s osou procházející gripperem
			* Nápady:
				\begitems \style o
				* snímat body v poly rozvnoměrně
				* snímat body ve sloupci dle osy
				* snímat celé tlusté řádky
				\enditems
			* Přiřadit obrázky
		\enditems
		\Black

\secc Finding Mathematical Features from Depth Map

		\Green
		\begitems \style o
			* Nalezení a vyplivnutí bodů ke zpracování
		\enditems
		\Black
