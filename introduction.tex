\chap Introduction

\sec Motivations

This bachelor thesis is part of~Clothes Perception and Manipulation project (CloPeMa, 2012-2015) funded by~the~European Commision. CloPeMa is research project which aims to~advance the~state of~the~art in~the~autonomous perception and manipulation of fabrics, textiles and garments. The CLoPeMa robot will learn to manipulate, perceive and fold a~variety of textiles~\cite[clopema]. 
%This bachelor thesis describes the design of method of measurement and extraction of image features. 

\sec Goals

The whole CloPeMa project is based on the manipulation of clothes (garments). For these manipulations is very good to have some simplified dynamic physical model of real garment. This is usefull for example for simulating or for counting model collisions. For creating virtual model of garments is important to have their parameters. This thesis has goal to design method of measurement and extraction of image features for obtaining parameters which will lead to construct the model.

\sec The State of Art

The main sphere of using dynamic simulation of garment is computer graphic. These simulations are only for a realistic look, but not for real dynamic physical behavior of garment~\ref[choi2005research] (including modern metod of simulating like~\cite[bender2013adaptive, lee2013automatic]). Simulation of garment from real physical parameters deals e.g.~\cite[luible2008simulation]. \mind{Dopsat pár vět + alespoň dva články}.

In the science and industry exist several measuring techniques which is used to find elementary parameters of fabrics e.g. KESF, FAST, PLMS or FAMOUS. These techniques measure e.g. flexural rigidity, shear, surface, compression or tensile properties, but need tens of measurement equipments and process to acquire parameters process takes from a few minutes (FAMOUS) up to units of hours (KESF).~\cite[kawabata1982objective, stylios1991textile, minazio1995fast, stylios2005new].

While existing methods give excellent results and a very detailed description of substances, but do not tell us anything about the whole garment. Moreover, these methods are very slow and very expensive.

We therefore propose which parameters we will need for build a simple dynamic physical model and we propose easiest way these parameters is obtained. We think that for such a simplified model, the parameters are well estimated from a moving garment, for which this model we want to build. This movement will cause by the robot and we follow this movement by avaible equipment  of robot~(chap.~\ref[sec:workplace]), so we use the RGB camera and rangefinder.


\mind{odhadování parametrů z videa}~\cite[bhat2003estimating]
